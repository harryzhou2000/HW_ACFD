%!TEX program = xelatex
\documentclass[UTF8,zihao=5]{ctexart} %ctex包的article


\usepackage[hidelinks]{hyperref}%超链接,自动加到目录里面



\title{{\bfseries\rmfamily\Huge{高等计算流体力学\hspace{1em}\\第11次作业}}}
\author{周涵宇 2022310984}
\date{}

\usepackage[a4paper]{geometry}
\geometry{left=0.75in,right=0.75in,top=1in,bottom=1in}%纸张大小和页边距

\usepackage[
UseMSWordMultipleLineSpacing,
MSWordLineSpacingMultiple=1.5
]{zhlineskip}%office风格的行间距

\usepackage{fontspec}
\setmainfont{Times New Roman}
\setsansfont{Source Sans Pro}
\setmonofont{Latin Modern Mono}
\setCJKmainfont{SimSun}[AutoFakeBold=true]
% \setCJKmainfont{仿宋}[AutoFakeBold=true]
\setCJKsansfont{黑体}[AutoFakeBold=true]
\setCJKmonofont{DengXian}[AutoFakeBold=true]

\setCJKfamilyfont{kaiti}{楷体}
\newfontfamily\CM{Cambria Math}


% \usepackage{indentfirst} %不工作 怎样调整ctex的段首缩进大小呢

\usepackage{fancyhdr}
\pagestyle{fancy}
\lhead{}
\chead{}
\rhead{}
\lfoot{}
\cfoot{\thepage}
\rfoot{}
\renewcommand{\headrulewidth}{1pt} %改为0pt即可去掉页眉下面的横线
\renewcommand{\footrulewidth}{1pt} %改为0pt即可去掉页脚上面的横线
\setcounter{page}{1}


% \usepackage{bm}

\usepackage{amsmath,amsfonts}
\usepackage{array}
\usepackage{enumitem}
\usepackage{unicode-math}

% \usepackage{titlesec} % it subverts the ctex titles
\usepackage{titletoc}


% titles in toc:
\titlecontents{section}
              [2cm]
              {\sffamily\zihao{5}\mdseries}%
              {\contentslabel{3em}}%
              {}%
              {\titlerule*[0.5pc]{-}\contentspage\hspace*{1cm}}

\titlecontents{subsection}
              [3cm]
              {\rmfamily\mdseries\zihao{5}}%
              {\contentslabel{3em}}%
              {}%
              {\titlerule*[0.5pc]{-}\contentspage\hspace*{1cm}}

\titlecontents{subsubsection}
              [4cm]
              {\rmfamily\mdseries\zihao{5}}%
              {\contentslabel{3em}}%
              {}%
              {\titlerule*[0.5pc]{-}\contentspage\hspace*{1cm}}
\renewcommand*\contentsname{\hfill \sffamily\mdseries 目录 \hfill}

\ctexset{
    section={   
        % name={前面,后面},
        number={\arabic{section}.},
        format=\sffamily\raggedright\zihao{4}\mdseries,
        indent= {0em},
        aftername = \hspace{0.5em},
        beforeskip=1ex,
        afterskip=1ex
    },
    subsection={   
        % name={另一个前面,另一个后面},
        number={\arabic{section}.\arabic{subsection}.}, %如果只用一个数字而非1.1
        format=\rmfamily\raggedright\mdseries\zihao{5},%正体字体,不加粗,main字体,五号字
        indent = {2em}, %缩进
        aftername = \hspace{0.5em},
        beforeskip=1ex,
        afterskip=1ex
    },
    subsubsection={   
        % name={另一个前面,另一个后面},
        number={\arabic{section}.\arabic{subsection}.\arabic{subsubsection}.}, %默认的 1.1.1
        format=\rmfamily\raggedright\mdseries\zihao{5},%无衬线字体,加粗,sans字体,五号字
        indent = {2em}, %缩进
        aftername = \hspace{0.5em},  %名字和标题间插入字符(此处是空白)
        beforeskip=1ex, %空行
        afterskip=1ex
    }
}

\usepackage{float}
\usepackage{graphicx}
\usepackage{multirow}
\usepackage{multicol}
\usepackage{caption}
\usepackage{subcaption}
\usepackage{cite}


%part、section、subsection、subsubsection、paragraph、subparagraph
\newcommand{\bm}[1]{{\mathbf{#1}}}
\newcommand{\trans}[0]{^\mathrm{T}}
\newcommand{\tran}[1]{#1^\mathrm{T}}
\newcommand{\hermi}[0]{^\mathrm{H}}
\newcommand{\conj}[1]{\overline{#1}}
\newcommand*{\av}[1]{\left\langle{#1}\right\rangle}
\newcommand*{\avld}[1]{\frac{\overline{D}#1}{Dt}}
\newcommand*{\pd}[2]{\frac{\partial #1}{\partial #2}}
\newcommand*{\pdcd}[3]{\frac{\partial^2 #1}{\partial #2 \partial #3}}
\newcommand*{\inc}[0]{{\Delta}}

\newcommand*{\uu}[0]{\bm{u}}
\newcommand*{\vv}[0]{\bm{v}}
\newcommand*{\g}[0]{\bm{g}}
\newcommand*{\nb}[0]{{\nabla}}

\newcommand*{\mean}[1]{\overline{#1}}



\begin{document}

\maketitle

\section{WENO5有限体积方法测试}

WENO5一维情况下,即为在模板$S_2=\{i-2,i-1,i\}$,$S_1=\{i-1,i,i+1\}$,$S_0=\{i,i+1,i+2\}$
计算二次多项式重构后,根据WENO非线性权将多项式组合即可。
均匀网格下,模板上线性重构系数见\cite{cockburn1998essentially}。
非线性权方面,本文采用原始的WENO5方案:
$$
    \omega_r = \frac{\alpha_r}{\sum_r{\alpha_r}},\ \ \alpha_r = \frac{d_r}{(\epsilon+\beta_r)^2}
$$
其中$\epsilon$取$10^{-6}$。
间断探测器采用原始WENO方案\cite{cockburn1998essentially}:
$$
    \begin{aligned}
        \beta_0 & =
        \frac{13}{12}(\mean{u}_{i}-2\mean{u}_{i+1}+\mean{u}_{i+2})^2 +
        \frac{1}{4}(3\mean{u}_{i}-4\mean{u}_{i+1}+\mean{u}_{i+2})^2 \\
        \beta_0 & =
        \frac{13}{12}(\mean{u}_{i-1}-2\mean{u}_{i}+\mean{u}_{i+1})^2 +
        \frac{1}{4}(\mean{u}_{i-1}-\mean{u}_{i+1})^2                \\
        \beta_2 & =
        \frac{13}{12}(\mean{u}_{i}-2\mean{u}_{i-1}+\mean{u}_{i-2})^2 +
        \frac{1}{4}(3\mean{u}_{i}-4\mean{u}_{i-1}+\mean{u}_{i-2})^2 \\
    \end{aligned}
$$
均匀网格上,重构$u_{i+1/2}$的理想线性权为
$$
    d_0 = \frac{3}{10},\ d_1=\frac{6}{10},\ d_2 = \frac{1}{10}
$$

求解欧拉方程(量热完全气体,$\gamma=1.4$)时,本文直接对守恒变量进行重构,可得界面守恒变量$u_{i-1/2,L},u_{i-1/2,R}$。
则根据黎曼求解器
$$
    f_{i-1/2} = \mathcal{R}(u_{i-1/2,L},u_{i-1/2,R})
$$
给出通量,近似黎曼求解器为HLLC格式\cite{2013Riemann}。

因此,可得有限体积半离散格式:
$$
    \frac{d \overline{u}_i }{dt} = \frac{f_{i-1/2} - f_{i+1/2}}{\inc x}
$$
记为
$$
    \frac{d \overline{u}}{dt} = R(\overline{u})
$$

采用三阶TVD-RK方法\cite{gottlieb1998total}时间推进:
$$
    \begin{aligned}
        \overline{u}^{(1)} & = \overline{u}^n + {\inc t} R(\overline{u}^n)                                                        \\
        \overline{u}^{(2)} & = \frac{3}{4}\overline{u}^n +\frac{1}{4}\overline{u}^{(1)} + \frac{\inc t}{4} R(\overline{u}^{(1)})  \\
        \overline{u}^{n+1} & = \frac{1}{3}\overline{u}^n +\frac{2}{3}\overline{u}^{(2)} + \frac{2\inc t}{3} R(\overline{u}^{(2)}) \\
    \end{aligned}
$$

\subsection{激波管算例}

计算一维Riemann问题,初始条件为:
$$
    \left\{
    \begin{array}{ll}
        \{\rho,u,p\} = \{1,0,1\},       & x < 0    \\
        \{\rho,u,p\} = \{0.125,0,0.1\}, & x \geq 0 \\
    \end{array}
    \right.
$$
采用计算域$[-0.5,0.5]$,CFL数$0.5$,计算$t=0.24$不同网格的结果如下。

\begin{figure}[H]
    \centering
    \includegraphics[width=16cm]{p7_HLLC_N50.png}  %需调整
    \caption{$50$网格WENO5激波管结果}
\end{figure}

\begin{figure}[H]
    \centering
    \includegraphics[width=16cm]{p7_HLLC_N100.png}  %需调整
    \caption{$100$网格WENO5激波管结果}
\end{figure}

\begin{figure}[H]
    \centering
    \includegraphics[width=16cm]{p7_HLLC_N200.png}  %需调整
    \caption{$200$网格WENO5激波管结果}
\end{figure}

图中可见,激波管为黎曼问题$RCS$结构,不同的网格数下都基本无振荡。
由于采用的是守恒变量重构,而非在特征空间,造成激波后有一定的数值振荡,但相对较小且随着网格加密变小。
不同网格数下,激波都有约2单元宽度,接触间断都约有4单元宽度。
因此,空间上,随着网格加密,间断的尺度与网格尺度是同一量级减小的。

\subsection{Shu-Osher算例}

计算一维Shu-Osher问题,初始条件为:
$$
    \left\{
    \begin{array}{ll}
        \{\rho,u,p\} = \{3.857,2.629,10.333\},   & x < 1    \\
        \{\rho,u,p\} = \{1+0.2\sin(5(x-5))0,1\}, & x \geq 1 \\
    \end{array}
    \right.
$$
采用计算域$[0,10]$,CFL数$0.5$,计算$t=1.8$。
计算中同时与二阶TVD有限体积对比,采用Van-Leer坡度限制器,其余格式与WENO5一致。
计算中采用10000单元WENO5的结果为参考解作为对照。
不同网格的结果如下。

\begin{figure}[H]
    \centering
    \includegraphics[width=16cm]{SH_SUM0.png}  %需调整
    \caption{不同网格Shu-Osher问题密度分布}
\end{figure}

\begin{figure}[H]
    \centering
    \includegraphics[width=16cm]{SH_SUM1.png}  %需调整
    \caption{不同网格Shu-Osher问题密度分布,光滑区}
\end{figure}

\begin{figure}[H]
    \centering
    \includegraphics[width=16cm]{SH_SUM2.png}  %需调整
    \caption{不同网格Shu-Osher问题密度分布,激波}
\end{figure}

不同区域中,5000网格时与参考解差距都很小。
在800网格时,WENO5在光滑区已经给出了可以接受的解,
而TVD的峰值明显不足。在200、400网格上,TVD对光滑区抹平很严重,
而WENO5保持了较多的非单调性。N=400时WENO5的波形已经基本与参考解有相同趋势。

激波附近,TVD与WENO5的表现类似,都有效抑制了振荡。

总体而言,WENO5在间断附近有较好的单调性,没有明显数值振荡;
在光滑区的精度明显好于TVD格式,推测其有较好色散耗散特性。
WENO5在Shu-Osher问题中可通过较少的网格达到很高的计算精度。

\section{线性权}

本文从重构多项式角度考虑怎样求理想权。

考虑定理:有限体积重构中,对于无限连续可微函数$u(x)$,若已知单元均值
$\mean{u}_j,j=1,2,...$,在$i$单元附近若有多项式
$ur(x)_i$,使得其满足$k$-Exact特性,则$ur(x)_i-u(x)=O((x-x_i)^{k+1})$,即
$ur(x)_i$是一个有$k+1$阶精度的重构多项式。

证明是平凡的。考虑幂级数展开:
$$
u(x)=\sum_{l=0}^{k}{\frac{u^{(l)}(x_i)}{l!}(x-x_i)^l} + R(x)
$$
其中余项量级为$R(x)=O((x-x_i)^{k+1})$
上式取单元平均:
$$
\mean{u}_{i+m}=\sum_{l=0}^{k}{\frac{u^{(l)}(x_i)}{l!}\mean{[(x-x_i)^l]}_{i+m}} + \mean{R}_{i+m}
$$
设$m\in S$为模板集合,则
由于$k$-Exact特性,可知
$$
ur(x)_i = \sum_{l=0}^{k}{\frac{u^{(l)}(x_i)}{l!}(x-x_i)^l} + \sum_m{L_{i,m}(x)\mean{R}_{i+m}}
$$
其中$L_{i,m}$是单元$i$在$m$位置的(均值意义的)拉格朗日重构基函数,
满足$\mean{[L_{i,m}]}_{i+p} = \delta_{mp}$。这样,就可以得知
重构中的这部分多项式:
$$
Rr(x)_i = \sum_m{L_{i,m}(x)\mean{R}_{i+m}} = O((x-x_i)^{k+1})\max(L_{i,m})
=O((x-x_i)^{k+1})
$$
因此,重构余项为:
$$
u(x) - ur(x)_i = R(x) - Rr(x)_i = O((x-x_i)^{k+1})
$$
满足$k+1$阶精度。

事实上,多维情况下采用最小二乘方法重构时,余项的量级需要考虑重构矩阵的条件数有界即可得到
同样结论。这样不需要拉格朗日基函数的构造。

现在讨论均匀网格上WENO3中的线性权问题。已知,有子模版重构:
$$
\begin{aligned}
    ur_0(x) &= \mean{u}_i + (x-x_i) \frac{\mean{u}_{i+1}-\mean{u}_{i}}{h}\\
    ur_1(x) &= \mean{u}_i + (x-x_i) \frac{\mean{u}_{i}-\mean{u}_{i-1}}{h}\\
\end{aligned}
$$
这两个多项式都满足$k$-Exact特性。现在希望将其加权组合,
使得其恰好等于大模板上的一个$k$-Exact重构:
$$
ur(x) = 
\frac{\mean{u}_{i-1}+\mean{u}_{i+1}-2\mean{u}_{i}}{2}\left(\frac{x-x_i}{h}\right)^2
+
\frac{\mean{u}_{i+1}-\mean{u}_{i-1}}{2}\frac{x-x_i}{h}
+\frac{2\mean{u}_{i}-\mean{u}_{i-1}-\mean{u}_{i+1}}{24} + \mean{u}_i
$$
记$\frac{x-x_i}{h} = \xi$,则可将$ur_0-\mean{u}_i,ur_1-\mean{u}_i,ur-\mean{u}_i$
都写作$\mean{u}_{i\pm1}-\mean{u}_i$的形式;
随后将$\mean{u}_{i\pm1}-\mean{u}_i$表示为$ur_0,ur_1$和$\xi$的组合,代入即有
$$
ur = 
\frac{ur_0 - ur_1}{2}\xi
+
\frac{ur_0 + ur_1}{2}
+\frac{ur_1-ur_0}{24 \xi} 
$$
即,有线性权:
$$
\begin{aligned}
    dr_0(\xi) &= \frac{1}{2} + \frac{\xi}{2}  - \frac{1}{24\xi}\\
    dr_1(\xi) &= \frac{1}{2} - \frac{\xi}{2}  + \frac{1}{24\xi}\\
\end{aligned}
$$
因此,右侧重构值需要的线性权为$\xi=1/2$,即
$$
d_0=dr_0(\frac{1}{2})=\frac{2}{3}, d_1=dr_1(\frac{1}{2})=\frac{1}{3}
$$
同样
$$
\overline{d}_0=dr_0(-\frac{1}{2})=\frac{1}{3}, \overline{d}_1=dr_1(-\frac{1}{2})=\frac{2}{3}
$$

这样,线性权给出的重构值是二次$k$-Exact重构的结果,根据最开始的定理
即可知达到三阶精度。

事实上,通用的求解方式是,在$\xi = \xi_p$处,列出
$$
dr_0(\xi_p)ur_0(\xi_p) + dr_1(\xi_p)ur_1(\xi_p) = ur(\xi_p)
$$
随后由于上式应该对满足多项式分布的每一种$\{\mean{u}_i,\mean{u}_{i+1},\mean{u}_{i-1}\}$
都成立,则列出方程使得$\mean{u}_i,\mean{u}_{i-1},\mean{u}_{i+1}$的系数
都等于0即可。在一维情况下,这个方程在任意阶数下是适定的;
而高维情况下,保持k-Exact特性的充分条件,
则是对上式代入所有多项式分布要求其成立。
这些方程通常是欠定的,因此可以通过调整自由参数来获取正的线性权\cite{hu1999weighted}。

可见,WENO的线性权计算在非结构网格下是较为复杂的问题。
当然,根据\cite{liu2013robust}的归纳,
另一种常见的作法是,非结构网格上每个子模版都用足够高阶数来重构,
这样线性权可以任取而达到相应精度阶数,如\cite{friedrich1998weighted}。
可以认为,后一种非结构网格WENO方法仅仅是采用了WENO的间断探测器来进行
非线性加权,或者说是一种限制器。
后一种作法明显需要更大的模板和,但好处是可以得到一个统一的
多项式分布而非逐点计算。

% $$
% ur(x) = [a_0 + b_0(x-x_i)]ur_0(x) + [a_1 + b_1(x-x_i)]ur_1(x)
% $$
% 注意到加权是一个1次多项式分布,是因为我们希望$ur(x)$是一个二次多项式。
% 此处,$a_0,a_1,b_0,b_1$都与$\mean{u}_i,\mean{u}_{i-1},\mean{u}_{i+1}$是有关的。

% 出于对称性,容易发现$a_0=a_1,b_0=-b_1$,此时列出$i,i+1$的均值条件即可。
% 化简后为:
% $$
% \begin{aligned}
%     2\mean{u}_i a_0 + \frac{\mean{u}_{i+1} + \mean{u}_{i-1} + \mean{u}_{i}}{12} hb_0 &= \mean{u}_{i}\\
%     (2\mean{u}_i + \mean{u}_{i+1} - \mean{u}_{i-1}) a_0 
%     +\frac{13\left[
%         \mean{u}_{i+1} + \mean{u}_{i-1} + \mean{u}_{i}
%     \right]}{12} hb_0 
%     &= \mean{u}_{i+1}
% \end{aligned}
% $$
% 可得
% $$
% a_0 = \frac{13\mean{u}_{i} - \mean{u}_{i+1}}{
%     24\mean{u}_i - \mean{u}_{i+1} + \mean{u}_{i-1} } \\
% b_0 = \frac{2\mean{u}_{i} - 2\mean{u}_{i+1}}{
%     24\mean{u}_i - \mean{u}_{i+1} + \mean{u}_{i-1} }
% $$


\bibliography{refs}{}
\bibliographystyle{unsrt}


\section*{附录}

本文使用的计算代码都在
\href{https://github.com/harryzhou2000/HW_ACFD}{Github的Git Repo(点击前往)}。




















% \section{SECTION 节}

% 一个

% \subsection{SUBSECTION 小节}

% 示例

% \subsubsection{SUBSUBSECTION 小节节}

% 字体字号临时调整:
% {
%    \sffamily\bfseries\zihao{3} 哈哈哈哈哈 abcde %三号 sans系列字体(一开始设置的) 加粗
%    %只对大括号范围内的后面的字有用,在标题、题注里面同样
% }
% { 
%    \CJKfamily{kaiti}\zihao{5}\itshape 哈哈哈哈哈 abcde%三号 kaiti(一开始设置的, 斜体(英文有变)
%    %只对大括号范围内的后面的字有用,在标题、题注里面同样
% }

% 一大堆一大堆一大堆一大堆一大堆一大堆一大堆一大堆一大堆一大堆
% 一大堆一大堆一大堆一大堆一大堆一大堆一大堆一大堆一大堆一大堆一大堆一大堆
% 一大堆一大堆一大堆一大堆一大堆一大堆一大堆一大堆一大堆一大堆一大堆一大堆
% 一大堆一大堆一大堆一大堆一大堆一大堆一大堆一大堆一大堆一大堆一大堆一大堆

% \begin{center}
%     居中的什么乱七八糟东西
% \end{center}


% 一个列表:
% \begin{itemize}
%     \item asef
%     \item[\%] asdf
%     \item[\#] aaa
% \end{itemize}

% 一个有序列表:
% \begin{enumerate}
%     \item asef
%     \item[\%\%] asdf
%     \item aaa
% \end{enumerate}

% 一个嵌套列表,考虑缩进:
% \begin{enumerate}[itemindent=2em] %缩进
%     \item asef \par asaf 东西东西东西东西东西东西东西东西东西东西东西东西东西东西东西东西东西东西东西东西东西东西东西东西,
%           F不是不是不是不是不是不是不是不是不是不是不是不是不是不是不是
%           \begin{itemize}[itemindent=2em]  %缩进
%               \item lalala
%               \item mamama
%           \end{itemize}
%     \item asdf
%     \item aaa
% \end{enumerate}

% \section{SECTION}

% 图片排版:

% \begin{figure}[H]
%     \begin{minipage}[c]{0.45\linewidth}  %需调整
%         \centering
%         \includegraphics[width=8cm]{RAM_O2_4660.png}  %需调整
%         \caption{第一个图}
%         \label{fig:a}
%     \end{minipage}
%     \hfill %弹性长度
%     \begin{minipage}[c]{0.45\linewidth}  %需调整
%         \centering
%         \includegraphics[width=8cm]{RAM_O4_4660.png}  %需调整
%         \caption{第二个图}
%         \label{fig:b}
%     \end{minipage}
% \end{figure}

% figure的选项为“htbp”时,会自动浮动,是“H”则和文字顺序严格一些。

% \begin{figure}[H]
%     \begin{minipage}[c]{0.45\linewidth}  %需调整
%         \centering
%         \includegraphics[width=8cm]{RAM_O2_4660.png}  %需调整
%         \label{fig:x}
%     \end{minipage}
%     \hfill %弹性长度
%     \begin{minipage}[c]{0.45\linewidth}  %需调整
%         \centering
%         \includegraphics[width=8cm]{RAM_O4_4660.png}  %需调整
%         \label{fig:y}
%     \end{minipage}
%     \caption{第三个图}
% \end{figure}

% \begin{figure}[H]
%     \centering
%     \includegraphics[width=8cm]{RAM_O4_4660.png}  %需调整
%     \label{fig:c}
%     \caption{第四个图}
% \end{figure}



% \subsection{SUBSECTION}

% 关于怎么搞表格:

% \begin{table*}[htbp]
%     \footnotesize
%     \begin{center}
%         \caption{一端力矩载荷下的结果\fontsize{0pt}{2em}} %需要学习统一设置;0代表不变?
%         \label{表2}
%         \begin{tabular}{|c|c|c|c|c|c|c|}
%             \hline
%             节点数                              & 积分方案              & 单元数                & $h=1m$                & $h=0.1m$              & $h=0.05m$             & $h=0.01m$             \\
%             \hline
%             \multirow{6}{*}{2}                  & \multirow{3}{*}{精确} & 1                     & 4.235294117647059E-08 & 1.406250000000000E-06 & 2.862823061630218E-06 & 1.439654482924097E-05 \\
%             \cline{3-7}
%                                                 &                       & 10                    & 5.975103734439814E-08 & 4.235294117646719E-05 & 1.800000000000410E-04 & 1.406249999999849E-03 \\
%             \cline{3-7}
%                                                 &                       &
%             10000                               & 5.999999915514277E-08 & 5.999996622448291E-05 & 4.799989509752562E-04 & 5.999793702477535E-02                                                 \\
%             \cline{2-7}
%                                                 & \multirow{3}{*}{减缩} & 1                     & 6.000000000000001E-08 & 5.999999999999972E-05 & 4.799999999999911E-04 & 6.000000000003492E-02 \\
%             \cline{3-7}
%                                                 &                       & 10                    & 6.000000000000071E-08 & 5.999999999999142E-05 & 4.799999999995399E-04 & 5.999999999903294E-02 \\
%             \cline{3-7}
%                                                 &                       & 10000                 & 6.000000112649221E-08 & 5.999999234537814E-05 & 4.799997501925065E-04 & 6.000037607984510E-02 \\
%             \hline

%             \multirow{6}{*}{3}                  & \multirow{3}{*}{精确} & 1                     & 6.000000000000003E-08 & 6.000000000000202E-05 & 4.800000000000831E-04 & 6.000000000056749E-02 \\
%             \cline{3-7}
%                                                 &                       & 10                    & 5.999999999999932E-08 & 6.000000000004190E-05 & 4.800000000000206E-04 & 6.000000001613761E-02 \\
%             \cline{3-7}
%                                                 &                       & 10000                 & 6.000000013769874E-08 & 5.999989495410481E-05 & 4.799942099727246E-04 & 6.000263852944890E-02 \\
%             \cline{2-7}
%                                                 & \multirow{3}{*}{减缩} & 1                     & 6.000000000000002E-08 & 6.000000000000267E-05 & 4.800000000000754E-04 & 5.999999999989982E-02 \\
%             \cline{3-7}
%                                                 &                       & 10                    & 5.999999999999899E-08 & 5.999999999987338E-05 & 4.799999999947916E-04 & 5.999999998625345E-02 \\
%             \cline{3-7}
%                                                 &                       & 10000                 & 5.999999728157785E-08 & 5.999994914321980E-05 & 4.800008377474699E-04 & 5.999472246346305E-02 \\
%             \hline

%             \multicolumn{3}{|c|}{欧拉-伯努利解} & 6.000000000000000E-08 & 6.000000000000000E-05 & 4.800000000000000E-04 & 6.000000000000000E-02                                                 \\
%             \hline
%         \end{tabular}
%     \end{center}
% \end{table*}

% 多行、多列表格的示例,基本思想是,多列的那个东西放在多列的最上面一格,下面的行要用\&来空开,也就是\&的数目
% 和普通表格一样,是列数减一;
% 多列的部分的话,就是每行内的操作,相应的\&就少了,见最后一行。

% tabular的“|c|c|c|c|c|c|c|”,意思是,竖线-居中-竖线-居中-竖线……,可以选择省略一些竖线;
% 每行之间的hline,代表贯通的横线,cline是有范围的横线。

% \subsubsection{SUBSUBSECTION}

% newcommand可以用来定义新指令,似乎基本上就是字符串替换……不太懂,总之在公式里面可以用,
% 外面也经常用。






% 公式这么写:
% \begin{equation}
%     \begin{aligned}
%         \frac{aa(x^1+x^2)}{\sqrt{x^1x^2}}
%         \nabla\times\uu
%         = & u_{j;m}\g^m\times\g^j
%         =u_{j;m}\epsilon^{mjk}\g_k
%         =u_{j,m}\epsilon^{mjk}\g_k                           \\
%         = & \frac{1}{\sqrt{g}}\left|
%         \begin{matrix}
%             \g_1       & \g_2       & \g_3       \\
%             \partial_1 & \partial_2 & \partial_3 \\
%             u_1        & u_2        & u_3
%         \end{matrix}
%         \right|
%         =\frac{\sqrt{x^1x^2}}{aa(x^1+x^2)}
%         \left|
%         \begin{matrix}
%             \g_1                        & \g_2                        & \g_3       \\
%             \partial_1                  & \partial_2                  & \partial_3 \\
%             u^1\frac{a^2(x^1+x^2)}{x^1} & u^2\frac{a^2(x^1+x^2)}{x^2} & u^3
%         \end{matrix}
%         \right|                                              \\
%         = & \frac{\sqrt{x^1x^2}}{aa(x^1+x^2)}
%         [[\g_1\,\g_2\,\g_3]]
%         diag\left(
%         u^3_{,2}-u^2_{,3}\frac{a^2(x^1+x^2)}{x^2},\,
%         u^1_{,3}\frac{a^2(x^1+x^2)}{x^1}-u^3_{,1},\, \right. \\
%           & \left.
%         u^2_{,1}\frac{a^2(x^1+x^2)}{x^2}+u^2\frac{a^2}{x^2}
%         -
%         u^1_{,2}\frac{a^2(x^1+x^2)}{x^1}-u^1\frac{a^2}{x^1}
%         \right)                                              \\
%         = & \frac{\sqrt{x^1x^2}}{aa(x^1+x^2)}
%         [[\bm{e}_1\,\bm{e}_2\,\bm{e}_3]]
%         \left[\begin{array}{ccc} a & -a & 0\\ \frac{a\,x^{2}}{\sqrt{x^{1}\,x^{2}}} & \frac{a\,x^{1}}{\sqrt{x^{1}\,x^{2}}} & 0\\ 0 & 0 & 1 \end{array}\right]              \\
%           & diag\left(
%         u^3_{,2}-u^2_{,3}\frac{a^2(x^1+x^2)}{x^2},\,
%         u^1_{,3}\frac{a^2(x^1+x^2)}{x^1}-u^3_{,1},\, \right. \\
%           & \left.
%         u^2_{,1}\frac{a^2(x^1+x^2)}{x^2}+u^2\frac{a^2}{x^2}
%         -
%         u^1_{,2}\frac{a^2(x^1+x^2)}{x^1}-u^1\frac{a^2}{x^1}
%         \right)
%     \end{aligned}
%     \label{eq:curlu}
% \end{equation}

% 如果不想带编号的公式(或者图表),用 equation* 这种环境。

% 引用,如果是引用的图表,就用表\ref{表2},图\ref{fig:a}这种,代码里是用label定义的标签来引用,
% 编号是自动生成的。公式引用一般写成:\eqref{eq:curlu}。目前这些引用自动会有超链接,反正有那个包自动
% 好像就会有……呜呜呜也不知道是怎么做到的,先这么用吧。

% \paragraph{PARA}

% 引用文献用\\cite这些,要用bibtex,暂时不做。

% \subparagraph{SUBPARA}

\end{document}