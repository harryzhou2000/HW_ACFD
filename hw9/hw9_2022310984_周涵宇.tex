%!TEX program = xelatex
\documentclass[UTF8,zihao=5]{ctexart} %ctex包的article


\usepackage[hidelinks]{hyperref}%超链接,自动加到目录里面



\title{{\bfseries\rmfamily\Huge{高等计算流体力学\hspace{1em}\\第9次作业}}}
\author{周涵宇 2022310984}
\date{}

\usepackage[a4paper]{geometry}
\geometry{left=0.75in,right=0.75in,top=1in,bottom=1in}%纸张大小和页边距

\usepackage[
UseMSWordMultipleLineSpacing,
MSWordLineSpacingMultiple=1.5
]{zhlineskip}%office风格的行间距

\usepackage{fontspec}
\setmainfont{Times New Roman}
\setsansfont{Source Sans Pro}
\setmonofont{Latin Modern Mono}
\setCJKmainfont{SimSun}[AutoFakeBold=true]
% \setCJKmainfont{仿宋}[AutoFakeBold=true]
\setCJKsansfont{黑体}[AutoFakeBold=true]
\setCJKmonofont{DengXian}[AutoFakeBold=true]

\setCJKfamilyfont{kaiti}{楷体}
\newfontfamily\CM{Cambria Math}


% \usepackage{indentfirst} %不工作 怎样调整ctex的段首缩进大小呢

\usepackage{fancyhdr}
\pagestyle{fancy}
\lhead{}
\chead{}
\rhead{}
\lfoot{}
\cfoot{\thepage}
\rfoot{}
\renewcommand{\headrulewidth}{1pt} %改为0pt即可去掉页眉下面的横线
\renewcommand{\footrulewidth}{1pt} %改为0pt即可去掉页脚上面的横线
\setcounter{page}{1}


% \usepackage{bm}

\usepackage{amsmath,amsfonts}
\usepackage{array}
\usepackage{enumitem}
\usepackage{unicode-math}

% \usepackage{titlesec} % it subverts the ctex titles
\usepackage{titletoc}


% titles in toc:
\titlecontents{section}
              [2cm]
              {\sffamily\zihao{5}\mdseries}%
              {\contentslabel{3em}}%
              {}%
              {\titlerule*[0.5pc]{-}\contentspage\hspace*{1cm}}

\titlecontents{subsection}
              [3cm]
              {\rmfamily\mdseries\zihao{5}}%
              {\contentslabel{3em}}%
              {}%
              {\titlerule*[0.5pc]{-}\contentspage\hspace*{1cm}}

\titlecontents{subsubsection}
              [4cm]
              {\rmfamily\mdseries\zihao{5}}%
              {\contentslabel{3em}}%
              {}%
              {\titlerule*[0.5pc]{-}\contentspage\hspace*{1cm}}
\renewcommand*\contentsname{\hfill \sffamily\mdseries 目录 \hfill}

\ctexset{
    section={   
        % name={前面,后面},
        number={\arabic{section}.},
        format=\sffamily\raggedright\zihao{4}\mdseries,
        indent= {0em},
        aftername = \hspace{0.5em},
        beforeskip=1ex,
        afterskip=1ex
    },
    subsection={   
        % name={另一个前面,另一个后面},
        number={\arabic{section}.\arabic{subsection}.}, %如果只用一个数字而非1.1
        format=\rmfamily\raggedright\mdseries\zihao{5},%正体字体,不加粗,main字体,五号字
        indent = {2em}, %缩进
        aftername = \hspace{0.5em},
        beforeskip=1ex,
        afterskip=1ex
    },
    subsubsection={   
        % name={另一个前面,另一个后面},
        number={\arabic{section}.\arabic{subsection}.\arabic{subsubsection}.}, %默认的 1.1.1
        format=\rmfamily\raggedright\mdseries\zihao{5},%无衬线字体,加粗,sans字体,五号字
        indent = {2em}, %缩进
        aftername = \hspace{0.5em},  %名字和标题间插入字符(此处是空白)
        beforeskip=1ex, %空行
        afterskip=1ex
    }
}

\usepackage{float}
\usepackage{graphicx}
\usepackage{multirow}
\usepackage{multicol}
\usepackage{caption}
\usepackage{subcaption}
\usepackage{cite}


%part、section、subsection、subsubsection、paragraph、subparagraph
\newcommand{\bm}[1]{{\mathbf{#1}}}
\newcommand{\trans}[0]{^\mathrm{T}}
\newcommand{\tran}[1]{#1^\mathrm{T}}
\newcommand{\hermi}[0]{^\mathrm{H}}
\newcommand{\conj}[1]{\overline{#1}}
\newcommand*{\av}[1]{\left\langle{#1}\right\rangle}
\newcommand*{\avld}[1]{\frac{\overline{D}#1}{Dt}}
\newcommand*{\pd}[2]{\frac{\partial #1}{\partial #2}}
\newcommand*{\pdcd}[3]{\frac{\partial^2 #1}{\partial #2 \partial #3}}
\newcommand*{\inc}[0]{{\Delta}}

\newcommand*{\uu}[0]{\bm{u}}
\newcommand*{\vv}[0]{\bm{v}}
\newcommand*{\g}[0]{\bm{g}}
\newcommand*{\nb}[0]{{\nabla}}



\begin{document}

\maketitle

\section{计算问题与计算格式}

根据Toro的书\cite{2013Riemann}有关章节,计算以下1D欧拉方程Riemann问题,

\begin{table}[H]
    % \tiny
    \begin{center}
        \caption{黎曼问题设置}
        \begin{tabular}{|c|c|c|c|c|c|c|}
            \hline
            问题 & $\rho_L$ & $V_L$     & $p_L$   & $\rho_L$ & $V_L$     & $p_L$   \\
            \hline
            1    & 1        & 0.75      & 1       & 0.125    & 0         & 0.1     \\
            \hline
            2    & 1        & -2        & 0.4     & 1        & 2         & 0.4     \\
            \hline
            3    & 1        & 0         & 1000    & 1        & 0         & 0.01    \\
            \hline
            4    & 1        & 0         & 0.01    & 1        & 0         & 100     \\
            \hline
            5    & 5.99924  & 19.5975   & 460.894 & 5.99242  & -6.19633  & 49.0950 \\
            \hline
            6    & 1        & -19.59745 & 1000    & 1        & -19.59745 & 0.01    \\
            \hline
        \end{tabular}
    \end{center}
\end{table}

根据以上过程解得结果为:

\begin{table}[H]
    % \footnotesize
    \small
    \begin{center}
        \caption{精确解}
        \begin{tabular}{|c|c|c|c|c|c|c|c|c|c|}
            \hline
            问题 & 类型 & $p_m$    & $V_m$   & $\rho_{mL}$                   & ${\rho_{mR}}$ & $S_{LL}$                      & $S_{LR}$                      & $S_{RL}$                      & $S_{RR}$ \\
            \hline
            1    & RCS  & 0.4663   & 1.3609  & 0.5799                        & 0.3397        & -0.4332                       & 0.2999                        & \multicolumn{2}{|c|}{2.1532}             \\
            \hline
            2    & RNR  & 0.001894 & 0       & \multicolumn{2}{|c|}{0.02185} & -2.7483       & -0.3483                       & 0.3483                        & 2.7483                                   \\
            \hline
            3    & RCS  & 460.8938 & 19.5975 & 0.5751                        & 5.9992        & -37.4166                      & -13.8996                      & \multicolumn{2}{|c|}{23.5175}            \\
            \hline
            4    & SCR  & 46.0950  & -6.1963 & 5.9924                        & 0.5751        & \multicolumn{2}{|c|}{-7.4375} & 4.3966                        & 11.8322                                  \\
            \hline
            5    & SCS  & 1691.6   & 8.6898  & 14.2823                       & 31.0426       & \multicolumn{2}{|c|}{0.7896}  & \multicolumn{2}{|c|}{12.2508}                                            \\
            \hline
            6    & RCS  & 460.8938 & 0       & 0.5751                        & 5.9992        & -57.0140                      & -33.4971                      & \multicolumn{2}{|c|}{3.9201}             \\
            \hline
        \end{tabular}
    \end{center}
\end{table}

其中$S_{LR},S_{LL}$指的是左侧波的速度,
假如是中心膨胀波则是膨胀波两侧的速度;
假如是左行激波则相等为激波的速度。
右侧同理。

本文计算方法
采用TVD-MUSCL方法重构,即在均匀一维网格下,有:

$$
    U_{i-1/2,R}=U_i-\frac{\inc x}{2} U_{x,i}
$$
$$
    U_{i+1/2,L}=U_i+\frac{\inc x}{2} U_{x,i}
$$
其中坡度$U_{x,i}$是二阶重构而来,为:
$$
    U_{x,i} = TVD(U_{x,i-1/2},U_{x,i+1/2})
$$
其中有界面坡度:
$$
    U_{x,i-1/2}=\frac{U_i-U_{i-1}}{\inc x}
$$
以及TVD坡度限制函数:
$$
TVD(a,b)=(sign(a)+sign(b)) \frac{|ab|}{|a+b|+\epsilon_p}
$$
为van-Leer限制器的坡度形式。对双精度浮点$\epsilon_p$取$10^{-300}$。
实际计算时,采用特征限制方案,即:
$$
    U_{x,i} = R_{i}TVD(L_{i-1/2}U_{x,i-1/2},L_{i+1/2}U_{x,i+1/2})
$$
界面上的特征变换$L_{i-1/2}$采用均值的Roe平均给出。

这样,可给出有限体积半离散:
$$
\pd{U_i}{t}+\frac{F(U_{i-1/2,L},U_{i-1/2,R})-F(U_{i+1/2,L},U_{i+1/2,R})}{\inc x} = 0
$$
其中$F(U_L,U_R)$是界面通量,使用近似黎曼通量HLLC\cite{2013Riemann}格式。

以下计算的是,设间断为0处出现,记末时刻在最快的波运动到$\pm0.8$的计算问题,计算CFL数取0.5。



\section{数值结果}

\newcommand*{\figureOfARS}[3]{
    \begin{figure}[H]
        \centering
        \includegraphics[width=18cm]{p#1_#2_N#3.png}  %需调整
        \caption{问题#1 #3网格}
    \end{figure}
}

\subsection{TVD格式结果}

\figureOfARS{1}{HLLC}{100}

\figureOfARS{1}{HLLC}{300}

\figureOfARS{2}{HLLC}{100}

\figureOfARS{2}{HLLC}{300}


\figureOfARS{3}{HLLC}{100}

\figureOfARS{3}{HLLC}{300}


\figureOfARS{4}{HLLC}{100}

\figureOfARS{4}{HLLC}{300}


\figureOfARS{5}{HLLC}{100}

\figureOfARS{5}{HLLC}{300}


\figureOfARS{6}{HLLC}{100}

\figureOfARS{6}{HLLC}{300}

以上的不同结果中,加密网格后都与解析解匹配更好;
在强度较大的激波、膨胀波附近存在一定非单调性。


\subsection{TVD与一阶的对比}
为了更好体现二阶格式与一阶格式的差别(零阶重构),
在100网格上对比两者。

\figureOfARS{1}{SUM}{100}
\figureOfARS{2}{SUM}{100}
\figureOfARS{3}{SUM}{100}
\figureOfARS{4}{SUM}{100}
\figureOfARS{5}{SUM}{100}
\figureOfARS{6}{SUM}{100}

其中,问题1的对比可见,在跨越0点的膨胀波附近,一阶格式有明显的残留的间断,而
TVD格式与光滑解匹配很好。问题3、4、5中,可见对接触间断和激波间的峰值预测上,
TVD明显好于一阶格式;所有的运动激波附近,TVD的解对激波的分辨宽度都略小于
一阶格式。

强激波或者膨胀波附近非单调部分,TVD的预测比一阶格式更好,说明这些非单调解是由于更大的数值耗散造成的。

问题2中,网格加密后,中央低密度压力区的速度预测反而变差,说明低密度区的计算需要对黎曼求解器和重构过程进一步改进。



本文结果与书\cite{2013Riemann}保持相符。

\bibliography{refs}{}
\bibliographystyle{unsrt}


\section*{附录}

本文使用的计算代码都在
\href{https://github.com/harryzhou2000/HW_ACFD}{Github的Git Repo(点击前往)}。




















% \section{SECTION 节}

% 一个

% \subsection{SUBSECTION 小节}

% 示例

% \subsubsection{SUBSUBSECTION 小节节}

% 字体字号临时调整:
% {
%    \sffamily\bfseries\zihao{3} 哈哈哈哈哈 abcde %三号 sans系列字体(一开始设置的) 加粗
%    %只对大括号范围内的后面的字有用,在标题、题注里面同样
% }
% { 
%    \CJKfamily{kaiti}\zihao{5}\itshape 哈哈哈哈哈 abcde%三号 kaiti(一开始设置的, 斜体(英文有变)
%    %只对大括号范围内的后面的字有用,在标题、题注里面同样
% }

% 一大堆一大堆一大堆一大堆一大堆一大堆一大堆一大堆一大堆一大堆
% 一大堆一大堆一大堆一大堆一大堆一大堆一大堆一大堆一大堆一大堆一大堆一大堆
% 一大堆一大堆一大堆一大堆一大堆一大堆一大堆一大堆一大堆一大堆一大堆一大堆
% 一大堆一大堆一大堆一大堆一大堆一大堆一大堆一大堆一大堆一大堆一大堆一大堆

% \begin{center}
%     居中的什么乱七八糟东西
% \end{center}


% 一个列表:
% \begin{itemize}
%     \item asef
%     \item[\%] asdf
%     \item[\#] aaa
% \end{itemize}

% 一个有序列表:
% \begin{enumerate}
%     \item asef
%     \item[\%\%] asdf
%     \item aaa
% \end{enumerate}

% 一个嵌套列表,考虑缩进:
% \begin{enumerate}[itemindent=2em] %缩进
%     \item asef \par asaf 东西东西东西东西东西东西东西东西东西东西东西东西东西东西东西东西东西东西东西东西东西东西东西东西,
%           F不是不是不是不是不是不是不是不是不是不是不是不是不是不是不是
%           \begin{itemize}[itemindent=2em]  %缩进
%               \item lalala
%               \item mamama
%           \end{itemize}
%     \item asdf
%     \item aaa
% \end{enumerate}

% \section{SECTION}

% 图片排版:

% \begin{figure}[H]
%     \begin{minipage}[c]{0.45\linewidth}  %需调整
%         \centering
%         \includegraphics[width=8cm]{RAM_O2_4660.png}  %需调整
%         \caption{第一个图}
%         \label{fig:a}
%     \end{minipage}
%     \hfill %弹性长度
%     \begin{minipage}[c]{0.45\linewidth}  %需调整
%         \centering
%         \includegraphics[width=8cm]{RAM_O4_4660.png}  %需调整
%         \caption{第二个图}
%         \label{fig:b}
%     \end{minipage}
% \end{figure}

% figure的选项为“htbp”时,会自动浮动,是“H”则和文字顺序严格一些。

% \begin{figure}[H]
%     \begin{minipage}[c]{0.45\linewidth}  %需调整
%         \centering
%         \includegraphics[width=8cm]{RAM_O2_4660.png}  %需调整
%         \label{fig:x}
%     \end{minipage}
%     \hfill %弹性长度
%     \begin{minipage}[c]{0.45\linewidth}  %需调整
%         \centering
%         \includegraphics[width=8cm]{RAM_O4_4660.png}  %需调整
%         \label{fig:y}
%     \end{minipage}
%     \caption{第三个图}
% \end{figure}

% \begin{figure}[H]
%     \centering
%     \includegraphics[width=8cm]{RAM_O4_4660.png}  %需调整
%     \label{fig:c}
%     \caption{第四个图}
% \end{figure}



% \subsection{SUBSECTION}

% 关于怎么搞表格:

% \begin{table*}[htbp]
%     \footnotesize
%     \begin{center}
%         \caption{一端力矩载荷下的结果\fontsize{0pt}{2em}} %需要学习统一设置;0代表不变?
%         \label{表2}
%         \begin{tabular}{|c|c|c|c|c|c|c|}
%             \hline
%             节点数                              & 积分方案              & 单元数                & $h=1m$                & $h=0.1m$              & $h=0.05m$             & $h=0.01m$             \\
%             \hline
%             \multirow{6}{*}{2}                  & \multirow{3}{*}{精确} & 1                     & 4.235294117647059E-08 & 1.406250000000000E-06 & 2.862823061630218E-06 & 1.439654482924097E-05 \\
%             \cline{3-7}
%                                                 &                       & 10                    & 5.975103734439814E-08 & 4.235294117646719E-05 & 1.800000000000410E-04 & 1.406249999999849E-03 \\
%             \cline{3-7}
%                                                 &                       &
%             10000                               & 5.999999915514277E-08 & 5.999996622448291E-05 & 4.799989509752562E-04 & 5.999793702477535E-02                                                 \\
%             \cline{2-7}
%                                                 & \multirow{3}{*}{减缩} & 1                     & 6.000000000000001E-08 & 5.999999999999972E-05 & 4.799999999999911E-04 & 6.000000000003492E-02 \\
%             \cline{3-7}
%                                                 &                       & 10                    & 6.000000000000071E-08 & 5.999999999999142E-05 & 4.799999999995399E-04 & 5.999999999903294E-02 \\
%             \cline{3-7}
%                                                 &                       & 10000                 & 6.000000112649221E-08 & 5.999999234537814E-05 & 4.799997501925065E-04 & 6.000037607984510E-02 \\
%             \hline

%             \multirow{6}{*}{3}                  & \multirow{3}{*}{精确} & 1                     & 6.000000000000003E-08 & 6.000000000000202E-05 & 4.800000000000831E-04 & 6.000000000056749E-02 \\
%             \cline{3-7}
%                                                 &                       & 10                    & 5.999999999999932E-08 & 6.000000000004190E-05 & 4.800000000000206E-04 & 6.000000001613761E-02 \\
%             \cline{3-7}
%                                                 &                       & 10000                 & 6.000000013769874E-08 & 5.999989495410481E-05 & 4.799942099727246E-04 & 6.000263852944890E-02 \\
%             \cline{2-7}
%                                                 & \multirow{3}{*}{减缩} & 1                     & 6.000000000000002E-08 & 6.000000000000267E-05 & 4.800000000000754E-04 & 5.999999999989982E-02 \\
%             \cline{3-7}
%                                                 &                       & 10                    & 5.999999999999899E-08 & 5.999999999987338E-05 & 4.799999999947916E-04 & 5.999999998625345E-02 \\
%             \cline{3-7}
%                                                 &                       & 10000                 & 5.999999728157785E-08 & 5.999994914321980E-05 & 4.800008377474699E-04 & 5.999472246346305E-02 \\
%             \hline

%             \multicolumn{3}{|c|}{欧拉-伯努利解} & 6.000000000000000E-08 & 6.000000000000000E-05 & 4.800000000000000E-04 & 6.000000000000000E-02                                                 \\
%             \hline
%         \end{tabular}
%     \end{center}
% \end{table*}

% 多行、多列表格的示例,基本思想是,多列的那个东西放在多列的最上面一格,下面的行要用\&来空开,也就是\&的数目
% 和普通表格一样,是列数减一;
% 多列的部分的话,就是每行内的操作,相应的\&就少了,见最后一行。

% tabular的“|c|c|c|c|c|c|c|”,意思是,竖线-居中-竖线-居中-竖线……,可以选择省略一些竖线;
% 每行之间的hline,代表贯通的横线,cline是有范围的横线。

% \subsubsection{SUBSUBSECTION}

% newcommand可以用来定义新指令,似乎基本上就是字符串替换……不太懂,总之在公式里面可以用,
% 外面也经常用。






% 公式这么写:
% \begin{equation}
%     \begin{aligned}
%         \frac{aa(x^1+x^2)}{\sqrt{x^1x^2}}
%         \nabla\times\uu
%         = & u_{j;m}\g^m\times\g^j
%         =u_{j;m}\epsilon^{mjk}\g_k
%         =u_{j,m}\epsilon^{mjk}\g_k                           \\
%         = & \frac{1}{\sqrt{g}}\left|
%         \begin{matrix}
%             \g_1       & \g_2       & \g_3       \\
%             \partial_1 & \partial_2 & \partial_3 \\
%             u_1        & u_2        & u_3
%         \end{matrix}
%         \right|
%         =\frac{\sqrt{x^1x^2}}{aa(x^1+x^2)}
%         \left|
%         \begin{matrix}
%             \g_1                        & \g_2                        & \g_3       \\
%             \partial_1                  & \partial_2                  & \partial_3 \\
%             u^1\frac{a^2(x^1+x^2)}{x^1} & u^2\frac{a^2(x^1+x^2)}{x^2} & u^3
%         \end{matrix}
%         \right|                                              \\
%         = & \frac{\sqrt{x^1x^2}}{aa(x^1+x^2)}
%         [[\g_1\,\g_2\,\g_3]]
%         diag\left(
%         u^3_{,2}-u^2_{,3}\frac{a^2(x^1+x^2)}{x^2},\,
%         u^1_{,3}\frac{a^2(x^1+x^2)}{x^1}-u^3_{,1},\, \right. \\
%           & \left.
%         u^2_{,1}\frac{a^2(x^1+x^2)}{x^2}+u^2\frac{a^2}{x^2}
%         -
%         u^1_{,2}\frac{a^2(x^1+x^2)}{x^1}-u^1\frac{a^2}{x^1}
%         \right)                                              \\
%         = & \frac{\sqrt{x^1x^2}}{aa(x^1+x^2)}
%         [[\bm{e}_1\,\bm{e}_2\,\bm{e}_3]]
%         \left[\begin{array}{ccc} a & -a & 0\\ \frac{a\,x^{2}}{\sqrt{x^{1}\,x^{2}}} & \frac{a\,x^{1}}{\sqrt{x^{1}\,x^{2}}} & 0\\ 0 & 0 & 1 \end{array}\right]              \\
%           & diag\left(
%         u^3_{,2}-u^2_{,3}\frac{a^2(x^1+x^2)}{x^2},\,
%         u^1_{,3}\frac{a^2(x^1+x^2)}{x^1}-u^3_{,1},\, \right. \\
%           & \left.
%         u^2_{,1}\frac{a^2(x^1+x^2)}{x^2}+u^2\frac{a^2}{x^2}
%         -
%         u^1_{,2}\frac{a^2(x^1+x^2)}{x^1}-u^1\frac{a^2}{x^1}
%         \right)
%     \end{aligned}
%     \label{eq:curlu}
% \end{equation}

% 如果不想带编号的公式(或者图表),用 equation* 这种环境。

% 引用,如果是引用的图表,就用表\ref{表2},图\ref{fig:a}这种,代码里是用label定义的标签来引用,
% 编号是自动生成的。公式引用一般写成:\eqref{eq:curlu}。目前这些引用自动会有超链接,反正有那个包自动
% 好像就会有……呜呜呜也不知道是怎么做到的,先这么用吧。

% \paragraph{PARA}

% 引用文献用\\cite这些,要用bibtex,暂时不做。

% \subparagraph{SUBPARA}

\end{document}